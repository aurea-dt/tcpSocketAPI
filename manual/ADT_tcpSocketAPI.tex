\documentclass[a4paper,10pt]{article}
\usepackage{graphicx}
\usepackage{ucs}
\usepackage[utf8x]{inputenc} %instalar latex-ucs.
\usepackage[spanish]{babel}
\usepackage{listings}
\usepackage[colorlinks=true]{hyperref}

% Title Page
\title{ADT\_TCPSocket API\\ Interfaz de programación para conexiones TCP/IP en C++}
\author{Mario Chririnos Colunga\\ Áurea - Desarrollo Tecnológico}


\begin{document}
\maketitle

% \begin{abstract}
% 
% \end{abstract}

\tableofcontents
\section{Introducción}
En este documento se describe nuestra interfaz de programación de aplicaciones para conexiones TCP/IP utilizando \textbf{\href{http://en.wikipedia.org/wiki/Berkeley\_sockets}{BSD sockets}}. Se utiliza \href{http://library.gnome.org/devel/glib/}{glib} para monitorizar el estado del los puertos. El documento está estructurado de la siguiente manera: en la sección \S\ref{api} describe el funcionamiento de nuestro código, la sección \S\ref{ejemplo} describe detalles relacionados al ejemplo incluido con el código fuente y la sección \S\ref{notas} proporciona notas y comentarios finales sobre este documento. 

\section{ADT\_TCPSocket API}
\label{api}

\begin{figure}

  \lstset{language=c++}
  \lstset{commentstyle=\textit}
  \tiny
  \begin{lstlisting}[frame=trbl]{}
#ifndef ADT_TCPSOCKET_H
#define ADT_TCPSOCKET_H

#include <iostream>
#include <sys/types.h>
#include <sys/socket.h>
#include <netinet/in.h>
#include <netdb.h>
#include <cstdlib>
#include <cstdio>
#include <cstring>
#include <glib.h>
#include <arpa/inet.h>

using namespace std;
//------------------------------------------------------------------------------
#define	CONNQUEUE	1 // connection backlog queue
#define	BUFFERSIZE	255
//------------------------------------------------------------------------------
class ADT_TCPSocket
{
 private:
 	int rxid;
 	GIOChannel* serverChannel;
 	GIOChannel* clientChannel;
 
 	int serverSocketFd;
	struct sockaddr_in server_addr;
	
 	int clientSocketFd;
	struct sockaddr_in client_addr;
	
	int sendData(int fd, const unsigned char *data,  int length) const;
	int sendToServer(const unsigned char* data, unsigned int length) const;
 	
	virtual void onGetData(struct sockaddr_in client);
	
	static int listen_callback(GIOChannel *source, GIOCondition condition, void *data);
	static int tcp_callback(GIOChannel *source, GIOCondition condition, void *data);
	
 protected:
  	unsigned char* buffer;
  	int bufferLength;
  	
 public: 
 	int connectToServer(const char *_server, int portno);
 	int sendData(const unsigned char* data, unsigned int length) const;
 	int sendData(const char* data, unsigned int length) const;
 	int sendData(const char *address, int port, const char *data,  int length) const;
 	int sendData(const char *address, int port, const unsigned char *data,  int length) const;

 	ADT_TCPSocket(int port);
	~ADT_TCPSocket();
};
//------------------------------------------------------------------------------

  \end{lstlisting}
  \caption{ADT\_TCPSocket.h}
  \label{joystick.h}
\end{figure}
  \subsection{Requisitos}
    Para compilar código de esta API se requiere contar con el paquete \textbf{libglib2.0-dev}, agregar al compilador C++ las banderas obtenidas mediante: \texttt{pkg-config --cflags glib-2.0},   \texttt{pkg-config --libs glib-2.0} y contar con un núcleo de Linux versión 2.4 o superior.

  \subsection{Constructor}
    El constructor \texttt{ADT\_TCPSocket(int port)} abre un puerto TCP/IP para que un cliente pueda estableser una conexión con el objeto creado.
	
  \subsection{Conexión}
    La función miembro \texttt{int connectToServer(const char *server, int portno)} permite conectarse el puerto $portno$ en el servidor con dirección $server$. La función devuelve cero si la conexión fue exitosa.

  \subsection{Envío de datos}
    Para enviar datos a través de la conexión TCP/IP están disponibles dos funciones:

  \begin{itemize}
   \item   \texttt{int sendData(const unsigned char* data, unsigned int length)}\\
	    La función envía un arreglo de datos $data$ de longitud $length$. Esta función debe ser utilizada cuando exista una conexión entre un un cliente y un servidor; Cuando el programa actúa como cliente la conexión con el servidor es creada por medio de la función \texttt{connectToServer(.)}; Cuando el programa es usado como servidor los datos son enviados a la conexión creada con el cliente.

   \item  \texttt{int sendData(const char *address, int port, const char *data,  int length)}\\
	  Esta función crea una conexión en el puerto $port$ con el servidor $address$ y envía un arreglo de datos $data$ de longitud $length$. Al terminar de enviar los datos la conexión se cerrada.


  \end{itemize}

    %Para desconectar el puerto se utiliza la función \texttt{void disconnect()}.


\subsection{Eventos}
La API utiliza \href{http://library.gnome.org/devel/glib/}{glib} para monitorizar el estado del puerto TCP/IP y llamar a una función de retro llamada (\textit{callback}) cuando se genere un evento. Se debe iniciar un \textit{ g main loop o gtk main loop} en el programa principal para activar los eventos.

La función virtual miembro \texttt{virtual void onGetData(struct sockaddr\_in client)} es llamada cada vez que se reciben datos por el puerto TCP/IP, esta función puede ser sobreescrita por clases derivadas. 

\section{Ejemplo}
\label{ejemplo}
  Junto con el código de esta API se provee de un programa ejemplo para demostrar su funcionamiento.

  \subsection{Requisitos}
  Para compilar el programa ejemplo se requiere contar con el paquete \textbf{libgtk2.0-dev}. Las banderas de compilación se especifican en el archivo \textit{makefile} de este ejemplo.

\section{Notas}
\label{notas}
El código fuente de esta API puede ser descargado en \href{http://www.aurea-dt.com/#tcpip}{nuestro sitio web}, en donde también se pueden reportar errores en el código fuente. Para reportar errores en este documento favor de escribir a errata@aurea-dt.com.



%\bibliographystyle{plain}	% (uses file "plain.bst")
%\bibliography{/home/mchc/Aurea/CPP/Programs/ADTlib-0.0.1/Manual/ADTlib.bib}
\end{document}          
